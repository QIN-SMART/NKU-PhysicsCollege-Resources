\documentclass[a4paper,12pt]{article}
\usepackage{fancyhdr}
\usepackage{geometry}
\usepackage{datetime2}
\geometry{left=2.5cm,right=2.5cm,top=2.5cm,bottom=2.5cm}
\usepackage{xeCJK}% filepath: /Users/qin/Desktop/experiment/report.tex
\setCJKmainfont{Songti SC}  
\setCJKfamilyfont{kai}{Kaiti SC}
\usepackage{amsmath,amssymb}
\usepackage{setspace} % 导言区添加
\usepackage{tabularx,diagbox,booktabs,multirow}
% 全文统一1.5倍行距
\onehalfspacing

\pagestyle{fancy}
\fancyhf{}
\fancyhead[L]{{\CJKfamily{kai} XXX实验}}
\fancyhead[C]{{\CJKfamily{kai} 秦春泰 2413630}}
\fancyhead[R]{{\CJKfamily{kai} 2025-0-}}
\fancyfoot[C]{{\CJKfamily{kai} \thepage}}



\begin{document}

\begin{center}
    {\LARGE \textbf{XXX实验报告}}\\[1.5em]
    姓名:秦春泰 \quad 学号:2413630 \quad 专业:物理学 \\[0.5em]
    组别:I \quad 实验时间:周五上午2025年9月26日
\end{center}

\vspace{1.5em}

\section{目的要求}
\begin{enumerate}
  \item 
  \item 
  \item 
\end{enumerate}

\section{仪器用具}
\begin{table}[htbp]
	\centering
	\renewcommand\arraystretch{1.6}
	% \setlength{\tabcolsep}{10mm}
	\begin{tabular}{p{0.10\textwidth}|p{0.20\textwidth}|p{0.10\textwidth}|p{0.10\textwidth}}
	\hline
	编号& 仪器用具名称 & 数量 &  参数 \\
	\hline
	1& &1 &  \\
	2& &1 &  \\
	3&  & 1 &  \\
	4& &1 &  \\
    5&  & 2 &  \\
    6&  & 1 &  \\
	\hline
\end{tabular}
\end{table}

\section{实验原理概述}
\subsection{}
\subsection{}
\subsection{}
\subsection{}

\section{实验内容与步骤}


\section{数据处理}
\subsection{实验数据}
\subsubsection{原始数据}
\subsubsection{处理数据}
\subsubsection{表格和图片}

\subsection{实验结果}
\subsection{误差分析}

\section{问题讨论}
对实验现象、数据误差、实验中遇到的问题等进行分析和讨论。

\section{思考题解答}
\begin{enumerate}
  \item 
  \item 
  \item 
\end{enumerate}

\end{document}